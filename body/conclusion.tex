% !TEX TS-program = XeLaTeX
% !TEX encoding = UTF-8 Unicode

\chapter*{\hfill 结  论 \hfill}
\addcontentsline{toc}{chapter}{结  论}
\label{conclusion}
\sloppy{}

简明扼要地回答研究问题:结论的首要目标是回答研究问题或验证研究假设。在结论部分,明确陈述你的研究问题,并提供对问题的回答或假设的验证结果。确保你的回答或验证是明确的,没有遗漏或模糊之处。

总结主要发现:总结你的研究中的主要发现和结果。列出你的研究结果,并将其归纳为几个主要观点或主题。确保提及每个主要发现,并指出其对研究问题的重要性和贡献。

讨论研究结果:在结论部分,可以对你的研究结果进行进一步的讨论。解释结果的意义和影响,以及与现有研究或理论的关联。探讨结果可能的解释、局限性和不确定性,并提出进一步研究的建议。

强调研究的重要性:在结论部分,再次强调你的研究的重要性和独特性。说明你的研究对学术界、实践领域或社会的意义和影响。指出你的研究填补了现有知识的空白或为进一步研究提供了基础。

提出建议或展望未来研究:根据你的研究结果,提出未来研究的建议或展望。指出你的研究可能存在的局限性,并提出进一步深入研究的方向。这样可以为其他研究者提供有关如何扩展你的研究的指导。

结论的凝练和清晰:结论部分应该简明扼要,避免冗长和重复。用清晰、简练的语言表达你的结论,并确保其与论文的整体内容保持一致。

结论是理论分析和实验结果的逻辑发展,是整篇论文的归宿。结论是在理论分析、试验结果的基础上,经过分析、推理、判断、归纳的过程而形成的总观点。结论必须完整、准确、鲜明、并突出与前人不同的新见解。

书写格式说明:

标题“结论”选用模板中的样式所定义的“结论”,或者手动设置成字体:黑体,居中,字号:小三,1.5倍行距,段后1行,段前为0行。

结论正文选用模板中的样式所定义的“正文”,每段落首行缩进2字;或者手动设置成每段落首行缩进2字,字体:宋体,字号:小四,行距:多倍行距 1.25,间距:段前、段后均为0行。