% !TEX TS-program = XeLaTeX
% !TEX encoding = UTF-8 Unicode

\chapter*{\hfill 修改记录 \hfill}
\addcontentsline{toc}{chapter}{修改记录}
\defaultfont
\linespread{1.25}
\sloppy{}

修改是论文写作过程中不可或缺的重要步骤,是提高论文质量的有效环节。修改的过程其实就是“去伪存真”、去糟粕取精华使论文不断“升华”的过程。

以下内容要求放到毕业设计(论文)修改记录中:

一、毕业设计(论文)题目修改

原题目:基于快速组网验证算法的优化算法

修稿后题目:基于 WiFi 和视觉的多模态行为识别方法研究

二、毕业设计(论文)内容重要修改记录

{\textbf {第一次修改记录:}}:

括号格式,修改前:中英文括号混用。

{\textbf{修改后:}}全部使用中文括号,在第一次出现的专有名词缩写后添加详细解释。

{\textbf {第二次修改记录:}}:

论文格式,修改前:列表换行没有顶头,公式变量没有统一。

{\textbf{修改后:}}修改列表格式,使用标准字母大小,全文统一变量指代内容,图标添加标注。

{\textbf {第三次修改记录:}}:

摘要和结论,修改前:内容较为混乱。

{\textbf{修改后:}}重写了摘要和结论部分,强调重要性和系统设计。

三、毕业设计(论文)外文翻译修改记录

四、毕业设计(论文)正式检测重复比

\hspace*{7.8cm}记录人(签字):\\
\hspace*{8.2cm}指导教师(签字):