% !TEX TS-program = XeLaTeX
% !TEX encoding = UTF-8 Unicode

\chapter{文献综述}
\label{chap01}
\defaultfont
\sloppy{}

\section{研究背景及意义}

\lipsum[1-2]

\section{国内外研究现状}

\textbf{引用样例}

文字\cite{khurana2018deep},多个引用样例\cite{khurana2018deep,baccouche2011sequential}。

\section{本文研究内容与结构}

\textbf{列表样例}

\begin{asparaenum}[(1)]
\item 本文提出了ViFi模型,对视频摄像头采集到的样本增加目标检测作为数据预处理的一部分,在单视频模态的训练上都有了不同程度的提升,测试结果准确率提高幅度从1.56\% 到 7.81\% 不等,使得多模态模型更加稳定,并在极端场景下有超过 20\% 的准确率提升。
\item 本文使用对时空域分别扩散卷积的时空域卷积网络替换了CRNN模型中对于Wi-Fi数据的处理,在单Wi-Fi模态的训练上提升幅度在 3.65\% 到 11.46\% 不等,对多模态模型的提升在不同遮挡的场景下提升约为2\%,不同光照的场景提升约为 1\%。
\item 本文尝试了多种模态融合方式(直接相联、相加、加权)并在不同场景下进行消融实验,并最终选定出最为稳定的融合方式。根据单模态的效果进行调整的权重加权融合通常会取得更为优秀的效果,直接相联相比最优加权的准确率降低幅度通常不超过1.5\%。
\item 本文最终将多种场景下的数据进行合并混杂训练,最终得到可以应对多种复杂场景的大型识别模型预训练权重,在特殊光线场景下准确率超过 93.75\%,在两种遮挡场景的识别准确率均为 95.833\%。
\end{asparaenum}

本文的剩余部分组织如下:第\ref{chap02}章介绍;第\ref{chap03}章介绍;第\ref{chap04}章介绍;第\ref{chap05}章介绍;最后一章对本文进行总结陈述。

% 正文是毕业设计(论文)的主体,是毕业论文或工程设计说明书的核心部分。要求学生运用所学的数学、自然科学、工程基础和专业知识解决复杂问题的能力,能够针对问题设计解决方案,在设计环节中体现创新意识,并考虑社会、健康、安全、法律、文化、环境以及社会可持续发展等因素;要着重反映毕业设计或论文的工作,要突出毕业设计的设计过程、设计依据及解决问题的方法;毕业论文重点要突出研究的新见解,例如新思想、新观点、新规律、新研究方法以及新结果等。

% 正文 (含引言或文献综述部分)内容应包括以下方面:

% 本研究内容的总体方案设计与选择论证;

% 本研究内容硬件与软件的设计计算,实验装置与测试方法等;

% 本研究内容试验方案设计的可行性、有效性、技术经济分析等,试验数据结果的处理与分析论证以及理论计算结果的分析与展望等;

% 本研究内容的理论分析。对本研究内容及成果应进行较全面、客观的理论阐述,应着重指出本研究内容中的创新、改进与实际应用。理论分析中,应将他人研究成果单独书写并注明出处,不得将其与本人提出的理论分析混淆在一起。对于将其他领域的理论、结果引用到本研究领域者,应说明该理论的出处,并论述引用的可行性与有效性。

% 自然科学的论文应推理正确,结论清晰,无科学性错误。

% 管理和人文学科的论文应包括对研究问题的论述和系统分析,比较研究,模型或方案设计,案例论证或实证分析,模型运行的结果或建议,改进措施等。

% 正文要求论点正确,推理严谨,数据可靠,文字精练,条理分明,文字图表规范、清晰和整齐,在论文的行文上,要注意语句通顺,达到科技论文所必须具备的“正确、准确、明确”的要求。计算单位采用国务院颁布的《统一公制计量单位中文名称方案》中规定和名称。各类单位、符号必须在论文中统一使用,外文字母必须注意大小写,正斜体。简化字采用正式公布过的,不能自造和误写。利用别人研究成果必须附加说明。引用前人材料必须引证原著文字。在论文的行文上,要注意语句通顺,达到科技论文所必须具备的“正确、准确、明确”的要求。

% \section{论文格式基本要求}

% 论文格式基本要求:

% (1) 纸  型:A4纸。

% (2) 打印要求:双面打印(除封面、任务书、原创性声明、关于使用授权的声明、中英文摘要等单面打印外,其余部分要求双面打印)。

% (3) 页边距:上3.5cm,下2.5cm,左2.5cm、右2.5cm。

% (4) 页  眉:2.5cm,页脚:2cm,左侧装订。

% (5) 字  体:正文全部宋体、小四。

% (6) 行  距:多倍行距:1.25,段前、段后均为0,取消网格对齐选项。

% \section{论文页眉页脚的编排}
% 一律用阿拉伯数字连续编页码。页码应由正文首页开始,作为第1页。封面不编入页码。将摘要、Abstract、目录等前置部分单独编排页码。页码必须标注在每页页脚底部居中位置,宋体,小五。

% 页眉,宋体,五号,居中。填写内容是“毕业设计(论文)中文题目”。

% 模板中已经将字体和字号要求自动设置为缺省值,只需双击页面中页眉位置,按要求将填写内容替换即可。

% \section{论文正文格式}
% 正文选用模板中的样式所定义的“正文”,每段落首行缩进2字;或者手动设置成每段落首行缩进2字,字体:宋体,字号:小四,行距:多倍行距 1.25,间距:段前、段后均为0行,取消网格对齐选项。

% 模板中已经自动设置为缺省值。

% 模板中的正文内容不具备自动调整格式的能力,如果要粘贴,请先粘贴在记事本编辑器中,再从记事本中拷贝,然后粘贴到正文中即可。或者使用手动设置,将粘贴内容的格式设置成要求的格式。

% \section{章节标题格式}
% (1) 每章的章标题选用模板中的样式所定义的“标题1”,居左;或者手动设置成字体:黑体,居左,字号:小三,1.5倍行距,段后11磅,段前为0。每章另起一页。章序号为阿拉伯数字。在输入章标题之后,按回车键,即可直接输入每章正文。

% (2) 每节的节标题选用模板中的样式所定义的“标题2”,居左;或者手动设置成字体:黑体,居左,字号:四号,1.5倍行距,段后为0,段前0.5行。

% (3) 节中的一级标题选用模板中的样式所定义的“标题3”,居左;或者手动设置成字体:黑体,居左,字号:小四,1.5倍行距,段后为0,段前0.5行。

% 正文各级标题编号的示例如图\ref{figure1.1}所示。
% \begin{figure}[htbp]
% 	\centering
% 	\includegraphics[scale = 0.4]{figures/1.1}
% 	\caption{\song\wuhao 标题编号示例}
% 	\label{figure1.1}
% \end{figure}

% \section{各章之间的分隔符设置}
% 各章之间应重新分页,使用“分页符”进行分隔。

% 设置方法:在“插入”菜单中选择“分隔符(B)…”,在弹出的窗口中选择分隔符类型为“分页符”,确定即可另起一页。

% \section{正文中的编号}
% 正文中的图、表、附注、公式一律采用阿拉伯数字分章编号。

% 如图1.2,表2.3,附注4.5,式6.7等。如“图1.2”就是指本论文第1章的第2个图。文中参考文献采用阿拉伯数字根据全文统一编号,如文献[3],文献[3,4],文献[6-10]等,在正文中引用时用右上角标标出。附录中的图、表、附注、参考文献、公式另行编号,如图A1,表B2,附注B3,或文献[A3]。


